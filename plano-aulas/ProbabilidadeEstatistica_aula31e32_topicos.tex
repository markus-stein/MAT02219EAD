% Options for packages loaded elsewhere
\PassOptionsToPackage{unicode}{hyperref}
\PassOptionsToPackage{hyphens}{url}
%
\documentclass[
]{article}
\usepackage{amsmath,amssymb}
\usepackage{lmodern}
\usepackage{iftex}
\ifPDFTeX
  \usepackage[T1]{fontenc}
  \usepackage[utf8]{inputenc}
  \usepackage{textcomp} % provide euro and other symbols
\else % if luatex or xetex
  \usepackage{unicode-math}
  \defaultfontfeatures{Scale=MatchLowercase}
  \defaultfontfeatures[\rmfamily]{Ligatures=TeX,Scale=1}
\fi
% Use upquote if available, for straight quotes in verbatim environments
\IfFileExists{upquote.sty}{\usepackage{upquote}}{}
\IfFileExists{microtype.sty}{% use microtype if available
  \usepackage[]{microtype}
  \UseMicrotypeSet[protrusion]{basicmath} % disable protrusion for tt fonts
}{}
\makeatletter
\@ifundefined{KOMAClassName}{% if non-KOMA class
  \IfFileExists{parskip.sty}{%
    \usepackage{parskip}
  }{% else
    \setlength{\parindent}{0pt}
    \setlength{\parskip}{6pt plus 2pt minus 1pt}}
}{% if KOMA class
  \KOMAoptions{parskip=half}}
\makeatother
\usepackage{xcolor}
\usepackage[margin=1in]{geometry}
\usepackage{graphicx}
\makeatletter
\def\maxwidth{\ifdim\Gin@nat@width>\linewidth\linewidth\else\Gin@nat@width\fi}
\def\maxheight{\ifdim\Gin@nat@height>\textheight\textheight\else\Gin@nat@height\fi}
\makeatother
% Scale images if necessary, so that they will not overflow the page
% margins by default, and it is still possible to overwrite the defaults
% using explicit options in \includegraphics[width, height, ...]{}
\setkeys{Gin}{width=\maxwidth,height=\maxheight,keepaspectratio}
% Set default figure placement to htbp
\makeatletter
\def\fps@figure{htbp}
\makeatother
\setlength{\emergencystretch}{3em} % prevent overfull lines
\providecommand{\tightlist}{%
  \setlength{\itemsep}{0pt}\setlength{\parskip}{0pt}}
\setcounter{secnumdepth}{-\maxdimen} % remove section numbering
\usepackage{fancyhdr}
\ifLuaTeX
  \usepackage{selnolig}  % disable illegal ligatures
\fi
\IfFileExists{bookmark.sty}{\usepackage{bookmark}}{\usepackage{hyperref}}
\IfFileExists{xurl.sty}{\usepackage{xurl}}{} % add URL line breaks if available
\urlstyle{same} % disable monospaced font for URLs
\hypersetup{
  pdftitle={Plano Aula 31 e 32},
  hidelinks,
  pdfcreator={LaTeX via pandoc}}

\title{Plano Aula 31 e 32}
\author{}
\date{}

\begin{document}
\maketitle

\addtolength{\headheight}{1.0cm}
\pagestyle{fancyplain} 
\lhead{\includegraphics[height=1.5cm]{logoIME.png}}
\rhead{\includegraphics[height=1.5cm]{logoEAD.png}}
\chead{UNIVERSIDADE FEDERAL DO RIO GRANDE DO SUL \\
INSTITUTO DE MATEMÁTICA E ESTATÍSTICA \\
DEPARTAMENTO DE ESTATÍSTICA \\
\vspace{0.3cm}
MAT02219 - Probabilidade e Estatística - 2022/1
}
\renewcommand{\headrulewidth}{0pt}

\hypertarget{peruxedodo-para-exame}{%
\section{Período para Exame}\label{peruxedodo-para-exame}}

Para quem ainda não atingiu média ponderada das avaliações (ler plano de
ensino) igual ou superior a \textbf{seis}.

\begin{itemize}
\tightlist
\item
  Revisar os principais conteúdos do semestre:
\end{itemize}

\hypertarget{avaliauxe7uxe3o-parcial-de-uxe1rea-1}{%
\subsubsection{Avaliação Parcial de Área
1}\label{avaliauxe7uxe3o-parcial-de-uxe1rea-1}}

\begin{itemize}
\tightlist
\item
  Introdução: população e amostra; tipos de estudos; amostragem;
\item
  Estatística Descritiva: distribuição de frequências;

  \begin{itemize}
  \tightlist
  \item
    tabelas e gráficos, normas;
  \item
    medidas descritivas e propriedades;
  \item
    análise exploratória, boxplot.
  \end{itemize}
\item
  Probabilidade: modelos determinísticos e probabilísticos;

  \begin{itemize}
  \tightlist
  \item
    experimento, espaco amostral, eventos e álgebra;
  \item
    definições de probabilidade, axiomas e propriedades;
  \item
    probabilidade condicional e independência;
  \item
    teoremas da probabilidade total e de Bayes.
  \end{itemize}
\end{itemize}

\vspace{1.0cm}

\hypertarget{avaliauxe7uxe3o-parcial-de-uxe1rea-2}{%
\subsubsection{Avaliação Parcial de Área
2}\label{avaliauxe7uxe3o-parcial-de-uxe1rea-2}}

\begin{itemize}
\tightlist
\item
  Variáveis Aleatórias: definição, discretas e contínuas;

  \begin{itemize}
  \tightlist
  \item
    distribuição de probabilidade e função de distribuição,
    propriedades;
  \item
    esperança e variância, propriedades;
  \item
    modelos de distribuição de probabilidade;
  \end{itemize}
\item
  Distribuição amostral: estatísticas e suas distribuições; teorema
  central do limite (TCL)
\item
  Inferências Estatística: objetivos e introdução

  \begin{itemize}
  \tightlist
  \item
    estimação pontual, estimador e estimativa; erro padrão;
  \end{itemize}
\item
  Intervalos de Confiaça:

  \begin{itemize}
  \tightlist
  \item
    para a uma média populacional \(\mu\), com variância conhecida ou
    desconhecida;
  \item
    para uma variância populacional \(\sigma^2\);
  \item
    para uma proporção populacional \(\pi\).
  \end{itemize}
\end{itemize}

\vspace{1.0cm}

\hypertarget{avaliauxe7uxe3o-parcial-da-uxe1rea-3}{%
\subsubsection{Avaliação Parcial da Área
3}\label{avaliauxe7uxe3o-parcial-da-uxe1rea-3}}

\begin{itemize}
\tightlist
\item
  Testes de hipóteses: definições e conceitos básicos, hipóteses;

  \begin{itemize}
  \tightlist
  \item
    tipos de erro, probabilidade de erros;
  \item
    estatistica de teste, região crítica, valor \emph{p};
  \end{itemize}
\item
  Testes para uma e duas médias populacionais;
\item
  Testes para proporções e variâncias;
\item
  Correlação Linear e Regressão linear simples.
\end{itemize}

\vspace{3.0cm}

\begin{itemize}
\item
  Refazer as avaliações.
\item
  Fazer o \textbf{simulado} para o exame.
\item
  Ter os \textbf{formulários} em mãos.
\end{itemize}

\vspace{1.0cm}

\begin{center}\rule{0.5\linewidth}{0.5pt}\end{center}

\hypertarget{boas-fuxe9rias-aos-juxe1-aprovadaos}{%
\subsubsection{Boas férias a(o)s já
aprovada(o)s!}\label{boas-fuxe9rias-aos-juxe1-aprovadaos}}

\hypertarget{bom-exame-aos-ainda-nuxe3o-aprovadaos}{%
\subsubsection{Bom exame a(o)s ainda não
aprovada(o)s!}\label{bom-exame-aos-ainda-nuxe3o-aprovadaos}}

\begin{center}\rule{0.5\linewidth}{0.5pt}\end{center}

\end{document}
