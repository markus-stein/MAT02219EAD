% Options for packages loaded elsewhere
\PassOptionsToPackage{unicode}{hyperref}
\PassOptionsToPackage{hyphens}{url}
%
\documentclass[
]{article}
\title{Plano Aula 19 e 20}
\author{}
\date{\vspace{-2.5em}}

\usepackage{amsmath,amssymb}
\usepackage{lmodern}
\usepackage{iftex}
\ifPDFTeX
  \usepackage[T1]{fontenc}
  \usepackage[utf8]{inputenc}
  \usepackage{textcomp} % provide euro and other symbols
\else % if luatex or xetex
  \usepackage{unicode-math}
  \defaultfontfeatures{Scale=MatchLowercase}
  \defaultfontfeatures[\rmfamily]{Ligatures=TeX,Scale=1}
\fi
% Use upquote if available, for straight quotes in verbatim environments
\IfFileExists{upquote.sty}{\usepackage{upquote}}{}
\IfFileExists{microtype.sty}{% use microtype if available
  \usepackage[]{microtype}
  \UseMicrotypeSet[protrusion]{basicmath} % disable protrusion for tt fonts
}{}
\makeatletter
\@ifundefined{KOMAClassName}{% if non-KOMA class
  \IfFileExists{parskip.sty}{%
    \usepackage{parskip}
  }{% else
    \setlength{\parindent}{0pt}
    \setlength{\parskip}{6pt plus 2pt minus 1pt}}
}{% if KOMA class
  \KOMAoptions{parskip=half}}
\makeatother
\usepackage{xcolor}
\IfFileExists{xurl.sty}{\usepackage{xurl}}{} % add URL line breaks if available
\IfFileExists{bookmark.sty}{\usepackage{bookmark}}{\usepackage{hyperref}}
\hypersetup{
  pdftitle={Plano Aula 19 e 20},
  hidelinks,
  pdfcreator={LaTeX via pandoc}}
\urlstyle{same} % disable monospaced font for URLs
\usepackage[margin=1in]{geometry}
\usepackage{graphicx}
\makeatletter
\def\maxwidth{\ifdim\Gin@nat@width>\linewidth\linewidth\else\Gin@nat@width\fi}
\def\maxheight{\ifdim\Gin@nat@height>\textheight\textheight\else\Gin@nat@height\fi}
\makeatother
% Scale images if necessary, so that they will not overflow the page
% margins by default, and it is still possible to overwrite the defaults
% using explicit options in \includegraphics[width, height, ...]{}
\setkeys{Gin}{width=\maxwidth,height=\maxheight,keepaspectratio}
% Set default figure placement to htbp
\makeatletter
\def\fps@figure{htbp}
\makeatother
\setlength{\emergencystretch}{3em} % prevent overfull lines
\providecommand{\tightlist}{%
  \setlength{\itemsep}{0pt}\setlength{\parskip}{0pt}}
\setcounter{secnumdepth}{-\maxdimen} % remove section numbering
\usepackage{fancyhdr}
\ifLuaTeX
  \usepackage{selnolig}  % disable illegal ligatures
\fi

\begin{document}
\maketitle

\addtolength{\headheight}{1.0cm}
\pagestyle{fancyplain} 
\lhead{\includegraphics[height=1.5cm]{logoIME.png}}
\rhead{\includegraphics[height=1.5cm]{logoEAD.png}}
\chead{UNIVERSIDADE FEDERAL DO RIO GRANDE DO SUL \\
INSTITUTO DE MATEMÁTICA E ESTATÍSTICA \\
DEPARTAMENTO DE ESTATÍSTICA \\
\vspace{0.3cm}
MAT02219 - Probabilidade e Estatística - 2021/2
}
\renewcommand{\headrulewidth}{0pt}

\hypertarget{avaliauxe7uxe3o-parcial-de-uxe1rea-2}{%
\section{Avaliação Parcial de Área
2}\label{avaliauxe7uxe3o-parcial-de-uxe1rea-2}}

Relembrar:

\begin{itemize}
\tightlist
\item
  Variáveis Aleatórias

  \begin{itemize}
  \tightlist
  \item
    definição, discretas e contínuas;
  \item
    distribuição de probabilidade e função de distribuição,
    propriedades;
  \item
    esperança e variância, propriedades;
  \item
    modelos de distribuição de probabilidade;
  \end{itemize}
\item
  Distribuição amostral

  \begin{itemize}
  \tightlist
  \item
    estatísticas e suas distribuições;
  \item
    teorema central do limite (TCL)
  \end{itemize}
\item
  Inferências Estatística

  \begin{itemize}
  \tightlist
  \item
    objetivos e introdução
  \item
    estimação pontual, estimador e estimativa;
  \item
    erro padrão
  \item
    estimação intervalar;
  \end{itemize}
\item
  Intervalos de Confiaça:

  \begin{itemize}
  \tightlist
  \item
    para a uma média populacional \(\mu\), com variância conhecida ou
    desconhecida;
  \item
    para uma variância populacional \(\sigma^2\);
  \item
    para uma proporção populacional \(\pi\).
  \end{itemize}
\end{itemize}

\vspace{0.5cm}

Usar o \textbf{formulário} para resolução das questões.

\vspace{0.5cm}

\textbf{Boa avaliação!!!}

\begin{center}\rule{0.5\linewidth}{0.5pt}\end{center}

\hypertarget{rever-planos-de-aula-slides-e-vuxeddeos.}{%
\subsection{Rever planos de aula, slides e
vídeos.}\label{rever-planos-de-aula-slides-e-vuxeddeos.}}

\hypertarget{fazer-o-simulado-para-a-prova-2---vale-ponto-extra}{%
\subsection{Fazer o simulado para a prova 2 - VALE PONTO
EXTRA!!!}\label{fazer-o-simulado-para-a-prova-2---vale-ponto-extra}}

\hypertarget{fazer-a-prova-2-atuxe9-o-dia-0401.}{%
\subsection{Fazer a prova 2 até o dia
04/01.}\label{fazer-a-prova-2-atuxe9-o-dia-0401.}}

\begin{center}\rule{0.5\linewidth}{0.5pt}\end{center}

\hypertarget{resumo-do-encontro-de-duxfavidas}{%
\subsection{Resumo do encontro de
dúvidas}\label{resumo-do-encontro-de-duxfavidas}}

\hypertarget{lista-2-2}{%
\subsubsection{Lista 2-2}\label{lista-2-2}}

Exercício \textbf{11(g)}

\begin{itemize}
\tightlist
\item
  Dados do problema:

  \begin{itemize}
  \tightlist
  \item
    \(X\): taxa de albumina no sangue em pessoas sadias ou normais;
  \item
    \(X \sim Normal(4,4; 0,6^2)\).
  \end{itemize}
\item
  No item (e) pede ``a taxa de albumina que é ultrapassada por 5\% da
  população''.

  \begin{itemize}
  \tightlist
  \item
    Ver o fórum de dúvidas da área 2, nosso monitor Andrei resolveu o
    item (e).
    \url{https://moodle.ufrgs.br/mod/forum/discuss.php?d=421833\#p2229541}
  \end{itemize}
\item
  No item (g) pede ``a taxa de albumina que não é ultrapassada por 10\%
  da população''.

  \begin{itemize}
  \tightlist
  \item
    precisamos encontrar um valor \(x^*\) da distribuição de \(X\) tal
    que a área abaixo dele seja?!
  \end{itemize}
\end{itemize}

\begin{figure}
\centering
\includegraphics{}
\caption{lista2-2ex11g.png}
\end{figure}

Assim segue o mesmo raciocínio do item (e)

\begin{figure}
\centering
\includegraphics{}
\caption{lista2-2ex11g2.png}
\end{figure}

\hypertarget{quiz-2-2}{%
\subsubsection{Quiz 2-2}\label{quiz-2-2}}

\begin{itemize}
\tightlist
\item
  Questão 1 e 2 necessitam encontrar valores da tabela \texttt{t} e
  qui-quadrado, porque em uma distribuição parece que procuramos valores
  unilaterais e em outra bilaterais, usamos \(\alpha\) ou \(\alpha/2\)?

  \begin{itemize}
  \tightlist
  \item
    No video \url{https://www.youtube.com/watch?v=v6BgWhAmV5I\&t=42s} a
    Profa. Lisiane fala dessa diferença nas tabelas que usamos.
  \item
    Importante é ter em mente que queremos encontrar valores de
    distribuições de probabilidade que deixam uma certa área acima ou
    abaixo deles.
  \end{itemize}
\item
  Questão 2 pede ``intervalo de XX\% de confiança para a variância
  populacional'';

  \begin{itemize}
  \tightlist
  \item
    O enunciado diz que a média populacional é desconhecida e fornece a
    média amostral e desvio padrão amostral;
  \item
    Nosso alvo aqui é o parâmetro \textbf{variância populacional}, então
    usaremos a estatística \textbf{variância amostral} para estimá-la;
  \item
    Nunca saberemos os valores dos \textbf{parâmetros} que são nosso
    alvo em inferência, assim usamos \textbf{estatísticas} e
    \textbf{intervalos de confiança} baseados nas estatísticas.
  \end{itemize}
\end{itemize}

\end{document}
